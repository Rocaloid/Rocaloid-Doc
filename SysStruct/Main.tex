\section{About Rocaloid}\indent

        Rocaloid is a free vocal synthesis \textbf{system}.
        
        Please notice the bold word ``\textbf{system}", which differs it from a ``software" or ``application": in a broad sense, Rocaloid provides a series of tools and libraries from the very basic DSP libs to synthesis engine and musical score editor. In a narrow sense, what Rocaloid represents in this reference is the back-end of this system.
        
        \bigskip
        
        \textbf{This reference focuses on Rocaloid in its narrow definition.}
        
        \bigskip

\subsection{Purpose}\indent

        Rocaloid aims at singing voice synthesis rather than speech synthesis(though it is possible to be used as a speech synthesizer).
        
        The description for history of Rocaloid Project can be found on our website:
        
        http://www.rocaloid.org
        
\subsection{Values}\indent

        In the back-end of Rocaloid, we code everything from scratch. The author Sleepwalking himself is a pure enthusiast in voice signal processing and he does not really care about development cycle. If you are looking for a reason for reinventing the wheel, we would suggest:
        
        \begin{itemize}
                \item Our wheels focus on audio signal processing. Surely they are more relevant to our topic.
                \item Our wheels are interrelated with great consistency. e.g. variable naming convention.
                \item Our wheels run faster than other wheels(on certain conditions and platforms which are commonly used by our users). e.g. CVEDSP performs 1024-point FFT 1.3 times faster than fftw3 on x86 with SSE.
                \item We intend to minimize dependencies of Rocaloid(libc only).
                \item The developers themselves learn from the project as they reinvent the wheels.
        \end{itemize}

\section{Current Status}\indent

        The development of Rocaloid3 started in September, 2013. The core synthesis engine, CVE3, was finished in December. However, a crucial defect was detected in the algorithm and we had to redesign the engine.
        
        Our decision was to abandon CVE3 and move on to CVE3.5, by the way restructuring the whole project.
        
        We have adopted SMS technique for the new synthesis engine. The detailed algorithm will be discussed in Chapter 2.
        
        Currently we are just about to start the restructuring.

\section{About the System}

\subsection{General Diagram}\indent

        From the users' point of view, Rocaloid reads in a music score file with melody and lyrics, accesses to a soundfont database, and outputs a .wav sound file.

        The soundfont is sampled from a real person(the sound provider). The raw samples undergo preprocesses(e.g. segmentation) and are then converted to sound font format.

\subsection{Inside Rocaloid}\indent

        After Rocaloid starts up, the music score file is first converted to a vocal description file. This file describes voice in a phonetic level(lower than muscial level). The second step is to convert the voice description file to a soundfont-specific engine script. Many details of voice are generated in this step, such as the exact duration for each phoneme.



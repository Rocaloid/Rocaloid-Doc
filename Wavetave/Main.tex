\section{About Wavetave}\indent

        Wavetave is an audio analysis and processing sandbox written in Octave Language and C++. The initiative of creating Wavetave is to test the feasibility of new algorithms, for example, the improved EpR Voice Model for CVE3.5.
        
        Before any new algorithm is put into application, it should be implemented and tested out in Wavetave first.
        
        \bigskip
        
        Wavetave has two parts: SpectrumVisualizer and MinCVE. The former is a powerful audio analyzer based on gnuplot; the later is a minimal and \textbf{very incomplete} version of CVE3.5, being developed just for testing the feasibility of CVE3.5.
        
\subsection{Source Code Quick Guide}\indent

        The code repository for Wavetave:
        
        https://github.com/Rocaloid/Wavetave
        
        \bigskip
        
        \begin{itemize}
                \item All source codes of Wavetave are placed under /src.
                \item SpectrumVisualizer and all of its plugins are placed under /src.
                \item MinCVE is placed under /src/MinCVE.
                \item Some useful code snippets are placed under /src/Util.
                \item All C++ written modules are placed under /src/Oct.
        \end{itemize}
        
        \subsubsection{Dependencies}\indent
        
        You need gnuplot, octave, octave-dev and g++ to compile and run Wavetave.
        
        \subsubsection{Building Wavetave}\indent
        
        \lstset{language = bash, tabsize = 4}
        
        \begin{lstlisting}
$ git clone https://github.com/Rocaloid/Wavetave
$ cd Wavetave/src/Oct
$ for i in *.cc; do mkoctfile $i -s; done #This may take a while
        \end{lstlisting}

\section{SpectrumVisualizer}
        
        \subsection{Basic Usage}\indent

        SpectrumVisualizer can be activated in two ways: direct execution or activating from an octave console.
        
        \lstset{language = bash, tabsize = 4}
        \begin{lstlisting}
$ cd Wavetave
$ ./SpectrumVisualizer.m
        \end{lstlisting}
        
        Or
        
        \lstset{language = octave, tabsize = 8}
        \begin{lstlisting}
octave:1> SpectrumVisualizer
        \end{lstlisting}
        
        \bigskip
        
        After starting up you will get two three windows: two plot windows and one terminal.
        
        Figure 1 is used to display the time-domain signal.
        
        Figure 2 displays the decibel-magnitude spectrum of the analyzed part of the signal.
        
        The terminal shows hints and extra information of the analyzed signal.
        
        \newpage
        
        \pic{11cm}{Wavetave/Startup.png}{SpectrumVisualizer starts up}
        
        When interacting with SpectrumVisualizer, you should always keep Figure 1 focused because it is the figure listening for mouse and key events.
        
        By default no wave is loaded when SpectrumVisualizer starts. Waves can be loaded by pressing O, then a prompt goes out in the terminal:
        
        \bigskip
        \texttt{Wave to open(enclosed by quotes):}
        \bigskip
        
        Input the file path with quotes. Currently SpectrumVisualizer only supports \textbf{single channel} wav files. The recommended sample rate is \textbf{44100Hz} and the recommended length is \textbf{1 to 5 seconds}. For example:
        
        \bigskip
        \texttt{Wave to open(enclosed by quotes):'/tmp/test.wav'}
        \bigskip
        
        If you are using gnome-terminal, this can be simply done by dragging the file into the terminal.

        Then press enter.
        
        Our example is a short pronunciation of `A' that lasts for 1 second.
        
        \newpage
        
        \pic{11cm}{Wavetave/Load.png}{`A'}
        
        Some plugins for Wavetave are activated by default, as shown in Figure 2 above. The red and green line and the labels come from Plugin\_FormantFitting. We are not sure about which plugins are set to be activated by default because the settings are constantly changed.
        
        \subsubsection{Surfing through the Wave}\indent
        
        You can change the position of the analysis window by clicking on Figure 1.
        
        Press W to scale in; Press S to scale out.
        
        Press A to move the window to left; Press D to move to right.
        
        Depending on the speed of your device and the number of plugins that are activated, the response speed of SpectrumVisualizer may vary a lot. When it appears to be too slow, try to disable some plugins.
        
        \subsubsection{Configuring the Plugins}\indent
        
        Open /src/SpectrumVisualizer.m with your favourite editor. Look for the following lines:
        
        \lstset{language = octave, tabsize = 8}
        \lstinputlisting{Wavetave/PluginConfig}
        
        Activate or deactivate a plugin by removing or adding the comment mark before it.
        
        \begin{itemize}
                \item Plugin\_Load stands for plugins that are called when a new wave file is loaded.
                \item Plugin\_Wave stands for plugins that are called when Figure 1 is repainted.
                \item Plugin\_Spectrum stands for plugins that are called when Figure 2 is repainted.
        \end{itemize}
        
        If you have implemented your own plugin, you can add its name to the above list. Make sure in each group of plugins there is at least an ``Empty" because it keeps the list from being empty.

\newpage

\subsection{Function Description}

\lstset{frame = none, numbers = none, language = octave, tabsize = 8}

\subsubsection{UpdateView}
        \begin{lstlisting}
function UpdateView(Wave)
        \end{lstlisting}
        
        Draws the time domain signal in the range of visible area.
        
        \fakepar{Parameters}
        
        \io
        {Wave: 1DArray & The whole wave signal.}
        {None &}
        
        \fakepar{Global Variables}
        
        \io
        {ViewPos: Scalar            & The center of analysis window \\ & & 
                                      in the loaded wave.\\
        & ViewWidth: Scalar         & The width of analysis window.\\
        & Length: Scalar            & The length of the loaded wave.\\
        & Plugin\_Wave: CharMatrix  & The list of names of plugin to \\ & &
                                      be called.
        }
        {PlotLeft: Scalar           & The left boundary of analysis \\ & &
                                      window in the loaded wave.
        }
        
        \fakepar{Behaviours}
        
        \begin{itemize}
                \item Figure 1 is replotted.
                \item Plugins in Plugin\_Wave are called.
        \end{itemize}

\subsubsection{UpdateSpectrum}
        \begin{lstlisting}
function [Ret, RetPhase, RetWave, ExtWave] = UpdateSpectrum(Wave)
        \end{lstlisting}
        
        Returns the spectrum and waveform in the range of FFT area.
        
        \fakepar{Parameters}
        
        \io
        {Wave: 1DArray              & The whole wave signal.}
        {Ret: 1DArray               & Decibel magnitude spectrum.\\
        & RetPhase: 1DArray         & Phase spectrum.\\
        & RetWave: 1DArray          & Waveform in analysis window.\\
        & ExtWave: 1DArray          & Waveform in analysis window.\\
        &                           & (128 samples longer than\\
        &                           & RetWave)}
        
        \fakepar{Global Variables}
        
        \io
        {ViewPos: Scalar            & The center of analysis window \\ & & 
                                      in the loaded wave.\\
        & ViewWidth: Scalar         & The width of analysis window.\\
        & Length: Scalar            & The length of the loaded wave.\\
        & FFTSize: Scalar           & Number of FFT points.
        }
        {None &}
        
        \fakepar{Behaviours}
        
        \begin{itemize}
                \item None.
        \end{itemize}




        
\section{Plugins}

        \subsection{Plugin\_F0Marking}\indent

        Corresponding to /src/Plugin\_F0Marking.m

\subsubsection{Usage}\indent

        This plugin finds an approximate fundamental frequency and labels it out in Figure 2.
        
        The fundamental frequency found is only an \textbf{approximation but not accurate}. The accuracy depends on frequency resolution of the spectrum.
        
        \pic{9cm}{Wavetave/PluginPicture/F0Marking.png}{Plugin\_F0Marking}
        
        \newpage

\subsection{Plugin\_F0Marking\_ByPhase}

\subsection{Plugin\_Freq2Pitch}

\subsection{Plugin\_HarmonicMarking}

\subsection{Plugin\_HarmonicMarking\_Naive}

\subsection{Plugin\_PhaseFigure}

\subsection{Plugin\_VOTMarking}

\subsection{Plugin\_Load\_PulseMarking}

\subsection{Plugin\_Load\_PulseMarking\_Naive}

\subsection{Plugin\_Load\_PulseMarking\_Stable}

\subsection{Plugin\_PulseMarking}

\subsection{Plugin\_UnvoicedDetection}

\subsection{Plugin\_Load\_EpRInitialization}

\subsection{Plugin\_FormantMarking\_EpR}

\subsection{Plugin\_FormantMarking\_Parabola}

\subsection{Plugin\_FormantFitting}

\subsection{PSOLAExtraction}

\subsection{PSOLASynthesis}

\subsection{EpR\_CumulateResonance}

\subsection{GenEstimateDiff}

\subsection{EpROptimize}

\subsection{ANTFit}




\section{MinCVE}

\subsection{Basic Usage}



\section{About Wavetave}\indent

        Wavetave is an audio analysis and processing sandbox written in Octave Language and C++. The initiative of creating Wavetave is to test the feasibility of new algorithms, for example, the improved EpR Voice Model for CVE3.5.
        
        Before any new algorithm is put into application, it should be implemented and tested out in Wavetave first.
        
        \bigskip
        
        Wavetave has two parts: SpectrumVisualizer and MinCVE. The former is a powerful audio analyzer based on gnuplot; the later is a minimal and \textbf{very incomplete} version of CVE3.5, being developed just for testing the feasibility of CVE3.5.
        
\subsection{Source Code Quick Guide}\indent

        The code repository for Wavetave:
        
        https://github.com/Rocaloid/Wavetave
        
        \bigskip
        
        \begin{itemize}
                \item All source codes of Wavetave are placed under /src.
                \item SpectrumVisualizer and all of its plugins are placed under /src.
                \item MinCVE is placed under /src/MinCVE.
                \item Some useful code snippets are placed under /src/Util.
                \item All C++ written modules are placed under /src/Oct.
        \end{itemize}
        
        \subsubsection{Dependencies}\indent
        
        You need gnuplot, octave, octave-dev and g++ to compile and run Wavetave.
        
        \subsubsection{Building Wavetave}\indent
        
        \lstset{language = bash, tabsize = 4}
        
        \begin{lstlisting}
$ git clone https://github.com/Rocaloid/Wavetave
$ cd Wavetave/src/Oct
$ for i in *.cc; do mkoctfile $i -s; done #This may take a while
        \end{lstlisting}

\section{SpectrumVisualizer}

\subsection{Basic Usage}

\section{Plugins}

\section{MinCVE}

\subsection{Basic Usage}


